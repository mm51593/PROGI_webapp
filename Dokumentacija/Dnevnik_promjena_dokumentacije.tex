\chapter{Dnevnik promjena dokumentacije}
		
		\textbf{\textit{Kontinuirano osvježavanje}}\\
				
		
		\begin{longtabu} to \textwidth {|X[2, l]|X[13, l]|X[3, l]|X[3, l]|}
			\hline \multicolumn{1}{|l|}{\textbf{Rev.}}	& \multicolumn{1}{l|}{\textbf{Opis promjene/dodatka}} & \multicolumn{1}{|l|}{\textbf{Autori}} & \multicolumn{1}{l|}{\textbf{Datum}} \\[3pt] \hline
			\endfirsthead
			
			\hline \multicolumn{1}{|l|}{\textbf{Rev.}}	& \multicolumn{1}{l|}{\textbf{Opis promjene/dodatka}} & \multicolumn{1}{|l|}{\textbf{Autori}} & \multicolumn{1}{l|}{\textbf{Datum}} \\[3pt] \hline
			\endhead
			
			\hline 
			\endlastfoot
			
			0.1 & Dodani dionici i aktori i opisano prvih 8 \textit{Use Case} dijagrama & Jukanović & 09.11.2020. \\[3pt] \hline 
			0.2 & Dodan dijagram obrasca uporabe (funkcionalnost prijavljenog korisnika) & Jukanović & 10.11.2020 \\[3pt] \hline 
			0.3 & Dodano još 7 (09-15) \textit{Use Case} dijagrama & Brstilo & 10.11.2020. \\[3pt] \hline
			0.4 & Modificirani zahtjevi aktora i dodani (16-25) \textit{Use case} dijagrama & Antunović & 10.11.2020. \\[3pt] \hline 
			0.5 & Dodan dijagram obrasca uporabe (funkcionalnost administratora) & Antunović & 11.11.2020. \\[3pt] \hline 
			0.6 & Dodan dijagram razreda generičke funkcionalnosti &  Cigula & 12.11.2020.\\[3pt] \hline 
			0.7 & Dodani sekvencijski dijagrami & Marošević & 13.11.2020. \\[3pt] \hline 
			0.8 & Dodana arhitektura baze podataka & Piškur, Marošević & 13.11.2020.  \\[3pt] \hline 
			0.9 & Dodani arhitektura sustava, opis projektnog zadatka i ostali zahtjevi & Rončević & 13.11.2020. \\[3pt] \hline 
			 &  &  &  \\[3pt] \hline 
			\textbf{1.0} & Korigiranje teksta i provjera dokumentacije & Marošević, Cigula, Jukanović & 13.11.2020 \\[3pt] \hline 
			 &  &  &  \\[3pt] \hline 
			 &  &  &  \\[3pt] \hline 
			 &  &  &  \\[3pt] \hline 
		  	 &  &  &  \\[3pt] \hline 
			 &  &  &  \\[3pt] \hline 
			 &  &  &  \\[3pt] \hline 
			 &  &  &  \\[3pt] \hline 
			 &  &  &  \\[3pt] \hline 
			 &  &  &  \\[3pt] \hline
			
			
		\end{longtabu}
	
	
		\textit{Moraju postojati glavne revizije dokumenata 1.0 i 2.0 na kraju prvog i drugog ciklusa. Između tih revizija mogu postojati manje revizije već prema tome kako se dokument bude nadopunjavao. Očekuje se da nakon svake značajnije promjene (dodatka, izmjene, uklanjanja dijelova teksta i popratnih grafičkih sadržaja) dokumenta se to zabilježi kao revizija. Npr., revizije unutar prvog ciklusa će imati oznake 0.1, 0.2, …, 0.9, 0.10, 0.11.. sve do konačne revizije prvog ciklusa 1.0. U drugom ciklusu se nastavlja s revizijama 1.1, 1.2, itd.}