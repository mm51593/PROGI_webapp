\chapter{Implementacija i korisničko sučelje}
		
		
		\section{Korištene tehnologije i alati}
		
			\textbf{\textit{dio 2. revizije}}
			
			 \textit{Detaljno navesti sve tehnologije i alate koji su primijenjeni pri izradi dokumentacije i aplikacije. Ukratko ih opisati, te navesti njihovo značenje i mjesto primjene. Za svaki navedeni alat i tehnologiju je potrebno \textbf{navesti internet poveznicu} gdje se mogu preuzeti ili više saznati o njima}.
			
			
			\eject 
		
	
		\section{Ispitivanje programskog rješenja}
			
			\textbf{\textit{dio 2. revizije}}\\
			
			 \textit{U ovom poglavlju je potrebno opisati provedbu ispitivanja implementiranih funkcionalnosti na razini komponenti i na razini cijelog sustava s prikazom odabranih ispitnih slučajeva. Studenti trebaju ispitati temeljnu funkcionalnost i rubne uvjete.}
	
			
			\subsection{Ispitivanje komponenti}
			
			\begin{figure}[H]
				\includegraphics[width=1\linewidth]{slike/unit_tests.PNG} %veličina u odnosu na širinu linije
				\caption{Ispitivanje jedinica}
				\label{fig:unit1} %label mora biti drugaciji za svaku sliku
			\end{figure}
		
			\begin{figure}[H]
				\includegraphics[width=1\linewidth]{slike/unit_test_result.PNG} %veličina u odnosu na širinu linije
				\caption{Rezultat ispitivanja jedinica}
				\label{fig:unit2} %label mora biti drugaciji za svaku sliku
			\end{figure}
			
			
			\subsection{Ispitivanje sustava}
			
			 \textit{Potrebno je provesti i opisati ispitivanje sustava koristeći radni okvir Selenium\footnote{\url{https://www.seleniumhq.org/}}. Razraditi \textbf{minimalno 4 ispitna slučaja} u kojima će se ispitati redovni slučajevi, rubni uvjeti te poziv funkcionalnosti koja nije implementirana/izaziva pogrešku kako bi se vidjelo na koji način sustav reagira kada nešto nije u potpunosti ostvareno. Ispitni slučaj se treba sastojati od ulaza (npr. korisničko ime i lozinka), očekivanog izlaza ili rezultata, koraka ispitivanja i dobivenog izlaza ili rezultata.\\ }
			 
			 \textit{Izradu ispitnih slučajeva pomoću radnog okvira Selenium moguće je provesti pomoću jednog od sljedeća dva alata:}
			 \begin{itemize}
			 	\item \textit{dodatak za preglednik \textbf{Selenium IDE} - snimanje korisnikovih akcija radi automatskog ponavljanja ispita	}
			 	\item \textit{\textbf{Selenium WebDriver} - podrška za pisanje ispita u jezicima Java, C\#, PHP koristeći posebno programsko sučelje.}
			 \end{itemize}
		 	\textit{Detalji o korištenju alata Selenium bit će prikazani na posebnom predavanju tijekom semestra.}
			
			\eject 
		
		
		\section{Dijagram razmještaja}
			
			 Dijagram razmještaja je strukturni statički UML dijagram koji opisuje topologiju sustava i usredotočen je na odnos sklopovskih i programskih dijelova. Postoji više vrsta dijagrama razmještaja, a ovdje se koristi specifikacijski dijagram razmještaja kako bi se prije navedeni odnos prikazao. Za ostvarenje ove aplikacije se koristi arhitektura "klijent - poslužitelj", a komunikacija se odvija protokolom HTTP. Na klijentskoj strani se nalazi web preglednik pomoću kojeg korisnik pristupa poslužitelju, odnosno šalje HTTP zahtjeve i prima HTTP odgovore. Poslužiteljska strana je nešto složenija te se ona sastoji od web poslužitelja na koji pristižu HTTP zahtjevi i poslužitelja baze podataka kojem web poslužitelj pristupa i iz kojeg uzima, uređuje ili sprema podatke.
			 
			 \begin{figure}[H]
			 	\includegraphics[width=1\linewidth]{slike/Dijagram_razmjestaja.PNG} %veličina u odnosu na širinu linije
			 	\caption{Dijagram razmještaja}
			 	\label{fig:dijraz} %label mora biti drugaciji za svaku sliku
			 \end{figure}
			
			\eject 
		
		\section{Upute za puštanje u pogon}
			\noindent\textbf{Heroku}\\
			Aplikacija je namjenjena puštanju u pogon preko platforme Heroku. Heroku je servis koji nudi usluge posluživanja aplikacija u više programskih jezika, među kojima su Ruby, Java, Node.js, Scala, Clojure, Python, PHP i Go, te usluge posluživanja baza podataka. Heroku je izabran za ovaj projekt zbog niske cijene (besplatan za stranicu ovog opsega) i jednostavnosti korištenja. Pretpostavit ćemo da korisnik na računalu ima instaliran Git te da je aplikacija preuzeta na računalo.
			\\\\
			\noindent\textbf{Stvaranje projekta}\\
			Potrebno je stvoriti račun na  \href{www.heroku.hr}{www.heroku.hr} te se prijaviti. Stranica će nas preusmjeriti na \textit{dashboard}, odakle stvaramo novi projekt klikom na gumb "New" - "Create new app" u gornjem desnom kutu grafičkog sučelja (Slika \ref{fig:newapp}). Pokazat će nam se forma za unos imena aplikacije i regije na kojoj će biti puštena u pogon (Slika \ref{fig:nameapp}). Nakon toga smo preusmjereni na stranicu novokreiranog projekta.
			\textbf{Napomena:} Ime aplikacije mora biti jedinstveno na razini čitave platforme. 
			\begin{figure}[H]
				\includegraphics[width=.9\linewidth]{slike/20210114_203519.png}
				\centering
				\caption{Nova aplikacija}
				\label{fig:newapp}
			\end{figure}
			\begin{figure}[H]
				\includegraphics[width=.9\linewidth]{slike/20210114_203555.png}
				\centering
				\caption{Ime aplikacije}
				\label{fig:nameapp}
			\end{figure}
			\noindent\textbf{Stvaranje baze podataka}\\
			Heroku pruža uslugu posluživanje PostgreSQL baze podataka. Kako bismo ju omogućili, na stranici projekta kliknemo na "Resources" karticu (Slika \ref{fig:resourcestab}). U traci za pretraživanje poglavlja "Add-ons" upišemo pojam "Heroku Postgres" i odaberemo prvi rezultat (Slika \ref{fig:herokupostgres}). Odaberemo plan pri vrhu sučelja, te kliknemo "Submit Order Form" (Slika \ref{fig:orderform}).\\
			Sada navigiramo na "Settings"  karticu (Slika \ref{fig:settingstab}) i nađemo poglavlje "Config Vars". Kliknemo gumb "Reveal Config Vars" za prikaz varijabli okruženja te radimo dvije stvari. Prvo se uvjerimo da postoji varijabla "DATABASE\_URL" koja je nastala kada smo stvorili PostgreSQL bazu. Nakon toga u polje "KEY" upisujemo pojam "SECRET" \textit{(bez navodnika)} i u polje "VALUE" upisujemo bilo kakav niz znakova. On će služiti za enkripciju poruka koje aplikacija razmjenjuje. Unos potvrdimo klikom na gumb "Add". Sučelje bi trebalo izgledati slično slici \ref{fig:configvars)}.
			\begin{figure}[H]
				\includegraphics[width=.9\linewidth]{slike/20210114_214638.png}
				\centering
				\caption{Kartica "Resources"}
				\label{fig:resourcestab}
			\end{figure}
			\begin{figure}[H]
				\includegraphics[width=.9\linewidth]{slike/20210114_215207.png}
				\centering
				\caption{Pretraživanje "Heroku Postgres"}
				\label{fig:herokupostgres}
			\end{figure}
			\begin{figure}[H]
				\includegraphics[width=.9\linewidth]{slike/20210114_214812.png}
				\centering
				\caption{"Submit Order Form"}
				\label{fig:orderform}
			\end{figure}
			\begin{figure}[H]
				\includegraphics[width=.9\linewidth]{slike/20210114_215739.png}
				\centering
				\caption{"Settings" kartica}
				\label{fig:settingstab}
			\end{figure}
			\begin{figure}[H]
				\includegraphics[width=.9\linewidth]{slike/20210114_220108.png}
				\centering
				\caption{Config Vars}
				\label{fig:configvars}
			\end{figure}
			\noindent\textbf{Heroku CLI}\\
			Heroku CLI \textit{(Command Line Interface)} nam omogućuje pristup platformi Heroku iz komandne linije. 
			Navigiramo nazad na karticu "Deploy". Kliknemo na poveznicu "Deployment method" - "Heroku Git" (Slika \ref{fig:herokucli)}) te pratimo uputstva za instalaciju ovisno o operacijskom sustavu koji koristimo.\\
			Po završetku instalacije otvorimo komandnu liniju \textit{cmd, terminal} i unesemo naredbu \code{heroku login}, nakon čega pratimo uputstva za prijavu.
			\begin{figure}[H]
				\includegraphics[width=.9\linewidth]{slike/20210114_205522.png}
				\centering
				\caption{Heroku CLI}
				\label{fig:herokucli}
			\end{figure}
			\noindent\textbf{Spajanje Git repozitorija i Heroku}\\
			U komandnoj liniji se pozicioniramo na direktorij aplikacije \textit{(naredba cd)}. Za idući korak je bitno da imamo inicijaliziran Git repozitoriji u direktoriju. To možemo provjeriti naredbom \code{git status}. Ukoliko je ispis \code{fatal: not a git repository...}, inicijaliziramo ga naredbom \code{git init}. Nakon toga dodajemo Heroku udaljeni repozitoriji pomoću \code{heroku git:remote -a <ime-aplikacije>}, gdje \textit{ime-aplikacije} zamijenimo imenom koji smo odabrali pri kreaciji projekta.\\
			Sada \textit{pushamo} projekt na Heroku udaljeni repozitorij slijedećim nizom naredbi:\\
			\code{git add .}\\
			\code{git commit -am "Commit"}\\
			\code{git push heroku master}\\
			Ovo može potrajati oko 1 minute. Nakon toga Heroku preuzima stvar i puštanje u pogon se automatizira.
			\\\\
			\noindent\textbf{Pristup stranici}\\
			Vraćamo se na Heroku web-stranicu. Jedino što preostaje je u gornjem desnom kutu \textit{dashboarda} kliknuti na "Open App" što će nas preusmjeriti na aplikaciju (Slika \ref{fig:openapp}).
			\begin{figure}[H]
				\includegraphics[width=.9\linewidth]{slike/20210114_222433.png}
				\centering
				\caption{Otvori aplikaciju}
				\label{fig:openapp}
			\end{figure}
			\eject 