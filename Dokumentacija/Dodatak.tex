\chapter*{Dodatak: Prikaz aktivnosti grupe}
		\addcontentsline{toc}{chapter}{Dodatak: Prikaz aktivnosti grupe}
		
		\section*{Dnevnik sastajanja}
		
		\textbf{\textit{Kontinuirano osvježavanje}}\\
		
		 \textit{U ovom dijelu potrebno je redovito osvježavati dnevnik sastajanja prema predlošku.}
		
		\begin{packed_enum}
			\item  sastanak
			
			\item[] \begin{packed_item}
				\item Datum: 14. listopada 2020.
				\item Prisustvovali: B.Antunović, T.Brstilo, L.Cigula,V.Jukanović, M.Marošević, M.Piskur, T.Rončević
				\item Teme sastanka:
				\begin{packed_item}
					\item  odabir objektno orijentiranog programskog jezika (Python 3)
					\item  odabir web frameworka (Flask)
					\item  rasprava vezana uz opće probleme izvođenja zadatka
					\item  postavljanje GitLab repozitorija
					\item  dogovor o strukturiranju repozitorija te interna pravila vezana uz commit
				\end{packed_item}
			\end{packed_item}
			
			\item  sastanak
			\item[] \begin{packed_item}
				\item Datum: 21. listopada 2020.
				\item Prisustvovali: B.Antunović, T.Brstilo, L.Cigula,V.Jukanović, M.Marošević, M.Piskur, T.Rončević
				\item Teme sastanka:
				\begin{packed_item}
					\item  odabir platforme za praćenje projekta i kooperaciju (Miro)
					\item  diskusija vezana uz interna pravila te način praćenja unutar alata
					\item  gruba podjela poslova na back-end i front-end
					\item  odabir template engine-a u okviru Flaska (Jinja2)
					\item  odabir DB toolkita u okviru Pythona (SQLAlchemy)
					\item  odabir register/login funkcionalnosti kao generičke funkcionalnosti koja će biti prezentirana nastavniku
				\end{packed_item}
			\end{packed_item}
			
			\item  sastanak
			\item[] \begin{packed_item}
				\item Datum: 28. listopada 2020.
				\item Prisustvovali: B.Antunović, T.Brstilo, L.Cigula,V.Jukanović, M.Marošević, M.Piskur, T.Rončević
				\item Teme sastanka:
				\begin{packed_item}
					\item  konkretna podjela poslova u okviru grube front-end, back-end podjele na 2. sastanku
					\item  podjela izrade dokumentacije među članovima tima
					\item  opća rasprava vezana uz opseg te zahtjevnost pojedinih poslova
					\item  postavljanje internih rokova u svrhu pravovremenog rješavanja kontinuiranih problema
					\item  diskusija vezana uz poteškoće povezivanja .html stranica i aplikacije
				\end{packed_item}
			\end{packed_item}			
			
			\item  sastanak
			\item[] \begin{packed_item}
				\item Datum: 4. studenoga 2020.
				\item Prisustvovali: B.Antunović, T.Brstilo, L.Cigula,V.Jukanović, M.Marošević, M.Piskur, T.Rončević
				\item Teme sastanka:
				\begin{packed_item}
					\item  rasprava vezana uz dijagrame razreda u Flasku
					\item  opća rasprava o pojedinim segmentima dokumentacije
					\item  dogovor vezan uz back-end implementaciju
					\item  opća rasprava o index stranici te već implementiranim rješenjima
				\end{packed_item}
			\end{packed_item}	
			
			%
			
		\end{packed_enum}
		
		\eject
		\section*{Tablica aktivnosti}
		
			\textbf{\textit{Kontinuirano osvježavanje}}\\
			
			 \textit{Napomena: Doprinose u aktivnostima treba navesti u satima po članovima grupe po aktivnosti.}
					
						
			
			\begin{longtabu} to \textwidth {|X[7, l]|X[1, c]|X[1, c]|X[1, c]|X[1, c]|X[1, c]|X[1, c]|X[1, c]|}
								
				\cline{2-8} \multicolumn{1}{c|}{\textbf{}} &     \multicolumn{1}{c|}{\rotatebox{90}{\textbf{Ime Prezime voditelja }}} & \multicolumn{1}{c|}{\rotatebox{90}{\textbf{Valentin Jukanović }}} &	\multicolumn{1}{c|}{\rotatebox{90}{\textbf{Ime Prezime }}} &	\multicolumn{1}{c|}{\rotatebox{90}{\textbf{Ime Prezime }}} &
				\multicolumn{1}{c|}{\rotatebox{90}{\textbf{Ime Prezime }}} &
				\multicolumn{1}{c|}{\rotatebox{90}{\textbf{Ime Prezime }}} &	\multicolumn{1}{c|}{\rotatebox{90}{\textbf{Ime Prezime }}} \\ \hline 
				\endfirsthead
				
			
				\cline{2-8} \multicolumn{1}{c|}{\textbf{}} &     \multicolumn{1}{c|}{\rotatebox{90}{\textbf{Ime Prezime voditelja}}} & \multicolumn{1}{c|}{\rotatebox{90}{\textbf{Valentin Jukanović }}} &	\multicolumn{1}{c|}{\rotatebox{90}{\textbf{Ime Prezime }}} &
				\multicolumn{1}{c|}{\rotatebox{90}{\textbf{Ime Prezime }}} &	\multicolumn{1}{c|}{\rotatebox{90}{\textbf{Ime Prezime }}} &
				\multicolumn{1}{c|}{\rotatebox{90}{\textbf{Ime Prezime }}} &	\multicolumn{1}{c|}{\rotatebox{90}{\textbf{Ime Prezime }}} \\ \hline 
				\endhead
				
				
				\endfoot
							
				 
				\endlastfoot
				
				Upravljanje projektom 		&  & 5 &  &  &  &  & \\ \hline
				Opis projektnog zadatka 	&  &  &  &  &  &  & \\ \hline
				
				Funkcionalni zahtjevi       &  & 2 &  &  &  &  &  \\ \hline
				Opis pojedinih obrazaca 	&  &  &  &  &  &  &  \\ \hline
				Dijagram obrazaca 			&  &  &  &  &  &  &  \\ \hline
				Sekvencijski dijagrami 		&  &  &  &  &  &  &  \\ \hline
				Opis ostalih zahtjeva 		&  &  &  &  &  &  &  \\ \hline

				Arhitektura i dizajn sustava	 &  &  &  &  &  &  &  \\ \hline
				Baza podataka				&  &  &  &  &  &  &   \\ \hline
				Dijagram razreda 			&  &  &  &  &  &  &   \\ \hline
				Dijagram stanja				&  &  &  &  &  &  &  \\ \hline
				Dijagram aktivnosti 		&  &  &  &  &  &  &  \\ \hline
				Dijagram komponenti			&  &  &  &  &  &  &  \\ \hline
				Korištene tehnologije i alati 		&  &  &  &  &  &  &  \\ \hline
				Ispitivanje programskog rješenja 	&  &  &  &  &  &  &  \\ \hline
				Dijagram razmještaja			&  &  &  &  &  &  &  \\ \hline
				Upute za puštanje u pogon 		&  &  &  &  &  &  &  \\ \hline 
				Dnevnik sastajanja 			&  &  &  &  &  &  &  \\ \hline
				Zaključak i budući rad 		&  &  &  &  &  &  &  \\  \hline
				Popis literature 			&  &  &  &  &  &  &  \\  \hline
				&  &  &  &  &  &  &  \\ \hline \hline
				\textit{izrada baze podataka} 		 			&  &  &  &  &  &  & \\ \hline 
				\textit{spajanje s bazom podataka} 							&  &  &  &  &  &  &  \\ \hline
				\textit{back end} 							&  &  &  &  &  &  &  \\  \hline
				 \textit{front end}							&  & 8 &  &  &  &  &\\  \hline
				
				
			\end{longtabu}
					
					
		\eject
		
		
		\textbf{\textit{dio 2. revizije}}\\
		
		\textit{Prenijeti dijagram pregleda promjena nad datotekama projekta. Potrebno je na kraju projekta generirane grafove s gitlaba prenijeti u ovo poglavlje dokumentacije. Dijagrami za vlastiti projekt se mogu preuzeti s gitlab.com stranice, u izborniku Repository, pritiskom na stavku Contributors.}
		
	