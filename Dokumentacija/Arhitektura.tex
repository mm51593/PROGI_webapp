\chapter{Arhitektura i dizajn sustava}
		
		\textbf{\textit{dio 1. revizije}}\\

		\textit{ Potrebno je opisati stil arhitekture te identificirati: podsustave, preslikavanje na radnu platformu, spremišta podataka, mrežne protokole, globalni upravljački tok i sklopovsko-programske zahtjeve. Po točkama razraditi i popratiti odgovarajućim skicama:}
	\begin{itemize}
		\item 	\textit{izbor arhitekture temeljem principa oblikovanja pokazanih na predavanjima (objasniti zašto ste baš odabrali takvu arhitekturu)}
		\item 	\textit{organizaciju sustava s najviše razine apstrakcije (npr. klijent-poslužitelj, baza podataka, datotečni sustav, grafičko sučelje)}
		\item 	\textit{organizaciju aplikacije (npr. slojevi frontend i backend, MVC arhitektura) }		
	\end{itemize}

	
		

		

				
		\section{Baza podataka}
			
			\textbf{\textit{dio 1. revizije}}\\
			
			\noindent Sutav koristi relacijsku bazu podataka. Baza se sastoji od relacija, tablica koje svojim atributima modeliraju pojave iz stvarnog svijeta. Uloga baze je strukurirana pohrana  podataka radi njihova lakšeg dohvaćanja i izmjenjivanja. Baza ovog sustava se sadrži od sljedećih entiteta:
			\begin{itemize}
				\item User (korisnik)
				\item Profile (osobni podaci korisnika)
				\item BillingInfo (podaci o plaćanju)
				\item Story (priča)
				\item StoryContent (dio sadržaja priče) 
				\item Model (maketa)
				\item ModelPhoto (fotografija koja se prilaže maketi)
				\item ModelPrice (cijena modela ovisno o materijalu)
				\item Order (narudžba)
				\item Order-Model (makete po narudžbi)
			\end{itemize}
		
			\subsection{Opis tablica}
			

				\noindent\textbf{User}  Podaci potrebni za prijavu u sustav. Sadrži atribute: identifikacijski broj, korisničko ime, e-mail adresu, lozinku i vrijeme nastanka računa. U vezi je \textit{One-to-One} s entitetom BillingInfo preko atributa id, u vezi \textit{One-to-One} s entitetom Profile preko atributa id, u vezi \textit{One-to-Many} s entitetom Story preko atributa id, u vezi \textit{One-to-Many} s entitetom Order preko atributa id te u vezi \textit{One-to-Many} s entitetom Model preko atributa it.
				
				\begin{longtabu} to \textwidth {|X[6, l]|X[6, l]|X[20, l]|}
					
					\hline \multicolumn{3}{|c|}{\textbf{User}}	 \\[3pt] \hline
					\endfirsthead
					
					\hline \multicolumn{3}{|c|}{\textbf{User}}	 \\[3pt] \hline
					\endhead
					
					\hline 
					\endlastfoot
					
					\cellcolor{LightGreen} id& INT	&  identifikacijski broj korisnika; samoinkrementirajući broj jedinstven za svakog korisnika	\\ \hline
					username& VARCHAR &  ime koje se prikazuje drugim korisnicima; jedinstveno 	\\ \hline 
					email & VARCHAR & E-mail adresa kojom se prijavljuje u sustav; jedinstvena  \\ \hline 
					password & CHAR & lozinka kojem se prijavljuje u sustav, kriptirana algoritmom Bcrypt na 60 bajtova \\ \hline
					time\_created & DATETIME	& vrijeme nastanka računa \\ \hline 

				\end{longtabu}
			
				\noindent\textbf{Profile} Podaci koje korisnik može prikazati na svom profilu. Sadrži atribute: ime, prezime, datum rođenja i kratak životopis. U vezi \textit{One-to-One} s entitetom User.
			
				\begin{longtabu} to \textwidth {|X[6, l]|X[6, l]|X[20, l]|}
				
					\hline \multicolumn{3}{|c|}{\textbf{Profile}}	 \\[3pt] \hline
					\endfirsthead
					
					\hline \multicolumn{3}{|c|}{\textbf{Profile}}	 \\[3pt] \hline
					\endhead
					
					\hline 
					\endlastfoot
					
					\cellcolor{LightGreen} user\_id& INT	&  identifikacijski broj korisnika kojemu profil pripada \\ \hline
					first\_name & VARCHAR &  korisnikovo ime; opcionalno	\\ \hline 
					last\_name & VARCHAR & korisnikovo prezime; opcionalno  \\ \hline 
					DOB & DATETIME & korisnikov datum rođenja; opcionalno \\ \hline
					bio & TEXT	& kratak životopis; opcionalno\\ \hline 
				
				\end{longtabu}
			
				\noindent\textbf{BillingInfo} Sadrži korisnikove podatke o plaćanju. Sadrži atribute: ime, prezime, adresa za naplatu, broj kreditne kartice, datum isteka kartice, CVC kartice. U vezi \textit{One-to-One} s entitetom User.
			
				\begin{longtabu} to \textwidth {|X[6, l]|X[6, l]|X[20, l]|}
	
					\hline \multicolumn{3}{|c|}{\textbf{BillingInfo}}	 \\[3pt] \hline
					\endfirsthead
					
					\hline \multicolumn{3}{|c|}{\textbf{BillingInfo}}	 \\[3pt] \hline
					\endhead
					
					\hline 
					\endlastfoot
					
					\cellcolor{LightGreen} user\_id& INT	&  identifikacijski broj korisnika kojemu podaci pripadaju \\ \hline
					first\_name & VARCHAR &  korisnikovo ime	\\ \hline 
					last\_name & VARCHAR & korisnikovo prezime  \\ \hline 
					billing\_address & DATETIME & adresa za naplatu \\ \hline
					card\_number & VARCHAR & broj kreditne kartice \\ \hline
					card\_expiry & DATETIME & datum isteka kartice \\ \hline
					card\_CVC & INT & CVC kartice \\ \hline
					
				\end{longtabu}
			
				\noindent\textbf{Story} Osnovne informacije o priči koju korisnik postavlja na aplikaciju. Sadrži atribute: identifikacijski broj, naslov priče, vrijeme nastanka, stanje (prihvaćena / neprihvaćena) i identifikacijski broj autora. U vezi je \textit{One-to-Many} s entitetom StoryContent te u vezi \textit{Many-to-One} s entitetom User.
			
				\begin{longtabu} to \textwidth {|X[6, l]|X[6, l]|X[20, l]|}
				
					\hline \multicolumn{3}{|c|}{\textbf{Story}}	 \\[3pt] \hline
					\endfirsthead
					
					\hline \multicolumn{3}{|c|}{\textbf{Story}}	 \\[3pt] \hline
					\endhead
					
					\hline 
					\endlastfoot
					
					\cellcolor{LightGreen} id& INT	&  identifikacijski broj priče; samoinkrementirajući broj jedinstven za svaku priču	\\ \hline
					title& TEXT &  naslov priče \\ \hline 
					time\_created & DATETIME & vrijeme nastanka priče \\ \hline 
					is\_approved & BOOLEAN & TRUE ako je priča prhvaćena, FALSE ako nije \\ \hline
					\cellcolor{LightBlue}author\_id & INT & identifikacijski broj autora \\ \hline 
				
				\end{longtabu}
			
				\noindent\textbf{StoryContent} Tekst, slika ili video koji se prilaže priči. Sadrži atribute: identifikacijski broj priče, redni broj dijela sadržaja, tekst priče, ime slike i ime videa.
				U vezi je \textit{Many-to-One} s entitetom Story.
				
				\begin{longtabu} to \textwidth {|X[6, l]|X[6, l]|X[20, l]|}
					
					\hline \multicolumn{3}{|c|}{\textbf{StoryContent}}	 \\[3pt] \hline
					\endfirsthead
					
					\hline \multicolumn{3}{|c|}{\textbf{StoryContent}}	 \\[3pt] \hline
					\endhead
					
					\hline 
					\endlastfoot
					
					\cellcolor{LightGreen} story\_id & INT &  identifikacijski broj priče kojemu se sadržaj prilaže \\ \hline
					\cellcolor{LightGreen}ordinal\_number & INT &  redni broj dijela sadržaja u priči	\\ \hline 
					story\_text & TEXT & tekst koji se prilaže priči; opcionalno \\ \hline 
					image\_name & VARCHAR & ime slike koja se prilaže priči; opcionalno \\ \hline
					video\_name & VARCHAR &  ime videa koji se prilaže priči, opcionalno \\ \hline 
					
				\end{longtabu}
			
				\noindent\textbf{Model} Podaci o maketi koju predlaže korisnik. Sadrži atribute: identifikacijski broj, ime, opis, identifikacijski broj korisnika koji ju predlaže. U vezi je \textit{Many-to-One} s entitetom User, u vezi \textit{One-to-Many} s entitetom ModelPrice, u vezi \textit{One-to-Many} s entitetom ModelPhoto te u vezi \textit{Many-to-Many} s entitetom Order-Model.
				
				\begin{longtabu} to \textwidth {|X[6, l]|X[6, l]|X[20, l]|}
					
					\hline \multicolumn{3}{|c|}{\textbf{Model}}	 \\[3pt] \hline
					\endfirsthead
					
					\hline \multicolumn{3}{|c|}{\textbf{Model}}	 \\[3pt] \hline
					\endhead
					
					\hline 
					\endlastfoot
					
					\cellcolor{LightGreen} id & INT &  identifikacijski broj makete; samoinkrementirajući broj jedinstven za svaku maketu \\ \hline
					name & VARCHAR &  ime makete; jedinstveno	\\ \hline 
					description & TEXT & opis makete \\ \hline 
					\cellcolor{LightBlue} creator\_id & INT & identifikacijski broj korisnika koji predlaže maketu \\ \hline
					
				\end{longtabu}
			
				\noindent\textbf{ModelPhoto} Fotografija koja se prilaže maketi. Sadrži atribute: ime slike i identifikacijski broj makete. U vezi \textit{Many-to-One} s entitetom Model.
				
				\begin{longtabu} to \textwidth {|X[6, l]|X[6, l]|X[20, l]|}
					
					\hline \multicolumn{3}{|c|}{\textbf{ModelPhoto}}	 \\[3pt] \hline
					\endfirsthead
					
					\hline \multicolumn{3}{|c|}{\textbf{ModelPhoto}}	 \\[3pt] \hline
					\endhead
					
					\hline 
					\endlastfoot
					
					\cellcolor{LightGreen} image\_name & VARCHAR & ime fotografije \\ \hline
					\cellcolor{LightBlue} model\_id & INT & identifikacijski broj makete kojemu se fotografija prilaže \\ \hline
					
				\end{longtabu}
			
				\noindent\textbf{ModelPrice} Cijena makete ovisno o materijalu od kojeg je izrađena. Sadrži atribute: identifijacijski broj modela, materijal od kojeg je izrađena i cijenu makete.
				U vezi \textit{Many-to-One} s entitetom Model.
				
				\begin{longtabu} to \textwidth {|X[6, l]|X[6, l]|X[20, l]|}
					
					\hline \multicolumn{3}{|c|}{\textbf{ModelPrice}}	 \\[3pt] \hline
					\endfirsthead
					
					\hline \multicolumn{3}{|c|}{\textbf{ModelPrice}}	 \\[3pt] \hline
					\endhead
					
					\hline 
					\endlastfoot
					
					\cellcolor{LightGreen} model\_id & INT & identifikacijski broj makete \\ \hline
					\cellcolor{LightGreen} material & VARCHAR & ime materijala \\ \hline
					price & NUMERIC & cijena makete za zadani materijal \\ \hline
					
				\end{longtabu}
			
				\noindent\textbf{Order} Podaci o narudžbi koju korisnik šalje trgovini. Sadrži atribute: identifikacijski broj narudžbe, vrijeme nastanka, identifikacijski broj korisnika koji šalje narudžbu. U vezi je \textit{Many-to-One} s entitetom User i u vezi \textit{One-to-Many} s entitetom Order-Model.
				
				\begin{longtabu} to \textwidth {|X[6, l]|X[6, l]|X[20, l]|}
					
					\hline \multicolumn{3}{|c|}{\textbf{Order}}	 \\[3pt] \hline
					\endfirsthead
					
					\hline \multicolumn{3}{|c|}{\textbf{Order}}	 \\[3pt] \hline
					\endhead
					
					\hline 
					\endlastfoot
					
					\cellcolor{LightGreen} id & INT &  identifikacijski broj narudžbe; samoinkrementirajući broj jedinstven za svaku narudžbu \\ \hline
					time\_created & DATETIME &  vrijeme nastanka	\\ \hline 
					\cellcolor{LightBlue} user\_id & INT & identifikacijski broj korisnika koji šalje narudžbu \\ \hline
					
				\end{longtabu}
			
				\noindent\textbf{Order-Model} Lista maketa u narudžbi. Sadrži atribute: identifikacijski broj narudžbe, identifikacijski broj makete, materijal od kojeg je maketa izrađena i cijena makete u trenutku narudžbe.
				
				\begin{longtabu} to \textwidth {|X[6, l]|X[6, l]|X[20, l]|}
					
					\hline \multicolumn{3}{|c|}{\textbf{Order-Model}}	 \\[3pt] \hline
					\endfirsthead
					
					\hline \multicolumn{3}{|c|}{\textbf{Order-Model}}	 \\[3pt] \hline
					\endhead
					
					\hline 
					\endlastfoot
					
					\cellcolor{LightBlue} order\_id & INT &  identifikacijski broj narudžbe \\ \hline
					\cellcolor{LightBlue} model\_id & INT & identifikacijski broj makete \\ \hline
					material & VARCHAR & ime materijala od kojeg je maketa izrađena \\ \hline
					price & NUMERIC & cijena makete za zadani materijal u trenutku narudžbe \\ \hline
					
				\end{longtabu}
			
			\subsection{Dijagram baze podataka}
				\begin{figure}[H]
					\includegraphics[width=.9\linewidth]{dijagrami/ER_baza.png}
					\caption{E-R dijagram baze podataka}
					\label{fig:erdija}
				\end{figure}
			\eject
			
			
		\section{Dijagram razreda}
		
			\begin{figure}[H]
						\includegraphics[width=.9\linewidth]{dijagrami/UML_class.PNG} %veličina u odnosu na širinu linije
						\caption{Dijagram razreda generičke funkcionalnosti}
						\label{fig:diraz1} %label mora biti drugaciji za svaku sliku
					\end{figure}
		
		\section{Dijagram stanja}
			
			
			\textbf{\textit{dio 2. revizije}}\\
			
			\textit{Potrebno je priložiti dijagram stanja i opisati ga. Dovoljan je jedan dijagram stanja koji prikazuje \textbf{značajan dio funkcionalnosti} sustava. Na primjer, stanja korisničkog sučelja i tijek korištenja neke ključne funkcionalnosti jesu značajan dio sustava, a registracija i prijava nisu. }
			
			
			\eject 
		
		\section{Dijagram aktivnosti}
			
			\textbf{\textit{dio 2. revizije}}\\
			
			 \textit{Potrebno je priložiti dijagram aktivnosti s pripadajućim opisom. Dijagram aktivnosti treba prikazivati značajan dio sustava.}
			
			\eject
		\section{Dijagram komponenti}
		
			\textbf{\textit{dio 2. revizije}}\\
		
			 \textit{Potrebno je priložiti dijagram komponenti s pripadajućim opisom. Dijagram komponenti treba prikazivati strukturu cijele aplikacije.}