\chapter{Arhitektura i dizajn sustava}
		
		\textbf{\textit{dio 1. revizije}}\\

		\noindent Arhitektura se moze podijeliti na tri podsustava: 
		\begin{itemize}
			\item Web poslužitelj
			\item Web aplikacija
			\item Baza podataka
		\end{itemize}
	
		\begin{figure}[H]
			\includegraphics[width=.9\linewidth]{slike/arhitektura_sustava.png}
			\centering
			\caption{Arhitektura sustava}
			\label{fig:arh1}
		\end{figure}

		\indent \underline{\textit{Web preglednik}} (internetski preglednik) program je koji omogućuje korisnicima pristup web stranicama i multimedijalnim sadržajima vezanih uz njih. Svaki internetski preglednik je prevoditelj što znači da je stranica pisana u kodu koji se potom interpretira kao nešto svakome razumljivo. Putem web preglednika korisnik šalje zahtjeve web poslužitelju. \\
		\indent \underline{\textit{Web poslužitelj}} je računalni program koji služi kao medij između korisnika i drugih programa ili uređaja. Poslužitelj je osnova rada web aplikacije budući da on pokreće web aplikaciju te joj preusmjerava zahtjeve klijenata. Njegova primarna funkcija je pohranjivanje, obrada i isporuka web stranica klijentima. Komunikacija izmedu klijenta i poslužitelja odvija se pomoću protokola HTTP (engl. \textit{Hyper Text Transfer Protocol}).\\
		\indent \underline{\textit{Web aplikacija}} je računalni program kojemu klijent pristupa iz web preglednika za obradu željenih zahtjeva. Ovisno o tipu zahtjeva, aplikacija pristupa bazi podataka zatim preko poslužitelja vraća korisniku odgovor vidljiv u web pregledniku kao HTML dokument.\\
		
		Programski jezik koji smo koristili za izradu web aplikacije "Maketa Shop" je Python s Flask radnim okvirom.\\
		
		MVC (Model-View-Controller) je obrazac koji se obično koristi za razvoj korisničkih sučelja zato jer dijeli programsku	logiku na tri medusobno povezana elementa što pojednostavnjuje razvoj web	aplikacija. Flask nije strukuriran kao čisti MVC okvir, ali po funkcionalnosti je vrlo sličan. Stoga možemo definirati sva tri elementa: 
		\begin{itemize}
			\item Model --- Središnja komponenta sustava.To je dinamička struktura podataka aplikacije neovisna o korisničkom sučelju. Izravno upravlja podacima, logikom i pravilima aplikacije i prima podatke od Controllera.
			\item View --- Svako predstavljanje informacija poput grafova, dijagrama ili tablica. Moguće je višestruko prikazivanje istih podataka, poput trake grafikona za upravljanje i tabličnog prikaza.
			\item Controller --- Prihvaća unos i pretvara ga u naredbe za Model ili View. Upravlja	korisničkim zahtjevima te izvodi daljnju interakciju.
		\end{itemize}
				
		\section{Baza podataka}
			
			\textbf{\textit{dio 1. revizije}}\\
			
			\noindent Sutav koristi relacijsku bazu podataka. Baza se sastoji od relacija, tablica koje svojim atributima modeliraju pojave iz stvarnog svijeta. Uloga baze je strukurirana pohrana  podataka radi njihova lakšeg dohvaćanja i izmjenjivanja. Baza ovog sustava se sadrži od sljedećih entiteta:
			\begin{itemize}
				\item User (korisnik)
				\item Profile (osobni podaci korisnika)
				\item BillingInfo (podaci o plaćanju)
				\item Story (priča)
				\item StoryContent (dio sadržaja priče) 
				\item Model (maketa)
				\item ModelPhoto (fotografija koja se prilaže maketi)
				\item ModelPrice (cijena modela ovisno o materijalu)
				\item Order (narudžba)
				\item Order-Model (makete po narudžbi)
			\end{itemize}
		
			\subsection{Opis tablica}
			

				\noindent\textbf{User}  Podaci potrebni za prijavu u sustav. Sadrži atribute: identifikacijski broj, korisničko ime, e-mail adresu, lozinku i vrijeme nastanka računa. U vezi je \textit{One-to-One} s entitetom BillingInfo preko atributa id, u vezi \textit{One-to-One} s entitetom Profile preko atributa id, u vezi \textit{One-to-Many} s entitetom Story preko atributa id, u vezi \textit{One-to-Many} s entitetom Order preko atributa id te u vezi \textit{One-to-Many} s entitetom Model preko atributa it.
				
				\begin{longtabu} to \textwidth {|X[6, l]|X[6, l]|X[20, l]|}
					
					\hline \multicolumn{3}{|c|}{\textbf{User}}	 \\[3pt] \hline
					\endfirsthead
					
					\hline \multicolumn{3}{|c|}{\textbf{User}}	 \\[3pt] \hline
					\endhead
					
					\hline 
					\endlastfoot
					
					\cellcolor{LightGreen} id& INT	&  identifikacijski broj korisnika; samoinkrementirajući broj jedinstven za svakog korisnika	\\ \hline
					username& VARCHAR &  ime koje se prikazuje drugim korisnicima; jedinstveno 	\\ \hline 
					email & VARCHAR & E-mail adresa kojom se prijavljuje u sustav; jedinstvena  \\ \hline 
					password & CHAR & lozinka kojem se prijavljuje u sustav, kriptirana algoritmom Bcrypt na 60 bajtova \\ \hline
					time\_created & DATETIME	& vrijeme nastanka računa \\ \hline 

				\end{longtabu}
			
				\noindent\textbf{Profile} Podaci koje korisnik može prikazati na svom profilu. Sadrži atribute: ime, prezime, datum rođenja i kratak životopis. U vezi \textit{One-to-One} s entitetom User.
			
				\begin{longtabu} to \textwidth {|X[6, l]|X[6, l]|X[20, l]|}
				
					\hline \multicolumn{3}{|c|}{\textbf{Profile}}	 \\[3pt] \hline
					\endfirsthead
					
					\hline \multicolumn{3}{|c|}{\textbf{Profile}}	 \\[3pt] \hline
					\endhead
					
					\hline 
					\endlastfoot
					
					\cellcolor{LightGreen} user\_id& INT	&  identifikacijski broj korisnika kojemu profil pripada \\ \hline
					first\_name & VARCHAR &  korisnikovo ime; opcionalno	\\ \hline 
					last\_name & VARCHAR & korisnikovo prezime; opcionalno  \\ \hline 
					DOB & DATETIME & korisnikov datum rođenja; opcionalno \\ \hline
					bio & TEXT	& kratak životopis; opcionalno\\ \hline 
				
				\end{longtabu}
			
				\noindent\textbf{BillingInfo} Sadrži korisnikove podatke o plaćanju. Sadrži atribute: ime, prezime, adresa za naplatu, broj kreditne kartice, datum isteka kartice, CVC kartice. U vezi \textit{One-to-One} s entitetom User.
			
				\begin{longtabu} to \textwidth {|X[6, l]|X[6, l]|X[20, l]|}
	
					\hline \multicolumn{3}{|c|}{\textbf{BillingInfo}}	 \\[3pt] \hline
					\endfirsthead
					
					\hline \multicolumn{3}{|c|}{\textbf{BillingInfo}}	 \\[3pt] \hline
					\endhead
					
					\hline 
					\endlastfoot
					
					\cellcolor{LightGreen} user\_id& INT	&  identifikacijski broj korisnika kojemu podaci pripadaju \\ \hline
					first\_name & VARCHAR &  korisnikovo ime	\\ \hline 
					last\_name & VARCHAR & korisnikovo prezime  \\ \hline 
					billing\_address & DATETIME & adresa za naplatu \\ \hline
					card\_number & VARCHAR & broj kreditne kartice \\ \hline
					card\_expiry & DATETIME & datum isteka kartice \\ \hline
					card\_CVC & INT & CVC kartice \\ \hline
					
				\end{longtabu}
			
				\noindent\textbf{Story} Osnovne informacije o priči koju korisnik postavlja na aplikaciju. Sadrži atribute: identifikacijski broj, naslov priče, vrijeme nastanka, stanje (prihvaćena / neprihvaćena) i identifikacijski broj autora. U vezi je \textit{One-to-Many} s entitetom StoryContent te u vezi \textit{Many-to-One} s entitetom User.
			
				\begin{longtabu} to \textwidth {|X[6, l]|X[6, l]|X[20, l]|}
				
					\hline \multicolumn{3}{|c|}{\textbf{Story}}	 \\[3pt] \hline
					\endfirsthead
					
					\hline \multicolumn{3}{|c|}{\textbf{Story}}	 \\[3pt] \hline
					\endhead
					
					\hline 
					\endlastfoot
					
					\cellcolor{LightGreen} id& INT	&  identifikacijski broj priče; samoinkrementirajući broj jedinstven za svaku priču	\\ \hline
					title& TEXT &  naslov priče \\ \hline 
					time\_created & DATETIME & vrijeme nastanka priče \\ \hline 
					is\_approved & BOOLEAN & TRUE ako je priča prhvaćena, FALSE ako nije \\ \hline
					\cellcolor{LightBlue}author\_id & INT & identifikacijski broj autora \\ \hline 
				
				\end{longtabu}
			
				\noindent\textbf{StoryContent} Tekst, slika ili video koji se prilaže priči. Sadrži atribute: identifikacijski broj priče, redni broj dijela sadržaja, tekst priče, ime slike i ime videa.
				U vezi je \textit{Many-to-One} s entitetom Story.
				
				\begin{longtabu} to \textwidth {|X[6, l]|X[6, l]|X[20, l]|}
					
					\hline \multicolumn{3}{|c|}{\textbf{StoryContent}}	 \\[3pt] \hline
					\endfirsthead
					
					\hline \multicolumn{3}{|c|}{\textbf{StoryContent}}	 \\[3pt] \hline
					\endhead
					
					\hline 
					\endlastfoot
					
					\cellcolor{LightGreen} story\_id & INT &  identifikacijski broj priče kojemu se sadržaj prilaže \\ \hline
					\cellcolor{LightGreen}ordinal\_number & INT &  redni broj dijela sadržaja u priči	\\ \hline 
					story\_text & TEXT & tekst koji se prilaže priči; opcionalno \\ \hline 
					image\_name & VARCHAR & ime slike koja se prilaže priči; opcionalno \\ \hline
					video\_name & VARCHAR &  ime videa koji se prilaže priči, opcionalno \\ \hline 
					
				\end{longtabu}
			
				\noindent\textbf{Model} Podaci o maketi koju predlaže korisnik. Sadrži atribute: identifikacijski broj, ime, opis, identifikacijski broj korisnika koji ju predlaže. U vezi je \textit{Many-to-One} s entitetom User, u vezi \textit{One-to-Many} s entitetom ModelPrice, u vezi \textit{One-to-Many} s entitetom ModelPhoto te u vezi \textit{Many-to-Many} s entitetom Order-Model.
				
				\begin{longtabu} to \textwidth {|X[6, l]|X[6, l]|X[20, l]|}
					
					\hline \multicolumn{3}{|c|}{\textbf{Model}}	 \\[3pt] \hline
					\endfirsthead
					
					\hline \multicolumn{3}{|c|}{\textbf{Model}}	 \\[3pt] \hline
					\endhead
					
					\hline 
					\endlastfoot
					
					\cellcolor{LightGreen} id & INT &  identifikacijski broj makete; samoinkrementirajući broj jedinstven za svaku maketu \\ \hline
					name & VARCHAR &  ime makete; jedinstveno	\\ \hline 
					description & TEXT & opis makete \\ \hline 
					\cellcolor{LightBlue} creator\_id & INT & identifikacijski broj korisnika koji predlaže maketu \\ \hline
					
				\end{longtabu}
			
				\noindent\textbf{ModelPhoto} Fotografija koja se prilaže maketi. Sadrži atribute: ime slike i identifikacijski broj makete. U vezi \textit{Many-to-One} s entitetom Model.
				
				\begin{longtabu} to \textwidth {|X[6, l]|X[6, l]|X[20, l]|}
					
					\hline \multicolumn{3}{|c|}{\textbf{ModelPhoto}}	 \\[3pt] \hline
					\endfirsthead
					
					\hline \multicolumn{3}{|c|}{\textbf{ModelPhoto}}	 \\[3pt] \hline
					\endhead
					
					\hline 
					\endlastfoot
					
					\cellcolor{LightGreen} image\_name & VARCHAR & ime fotografije \\ \hline
					\cellcolor{LightBlue} model\_id & INT & identifikacijski broj makete kojemu se fotografija prilaže \\ \hline
					
				\end{longtabu}
			
				\noindent\textbf{ModelPrice} Cijena makete ovisno o materijalu od kojeg je izrađena. Sadrži atribute: identifijacijski broj modela, materijal od kojeg je izrađena i cijenu makete.
				U vezi \textit{Many-to-One} s entitetom Model.
				
				\begin{longtabu} to \textwidth {|X[6, l]|X[6, l]|X[20, l]|}
					
					\hline \multicolumn{3}{|c|}{\textbf{ModelPrice}}	 \\[3pt] \hline
					\endfirsthead
					
					\hline \multicolumn{3}{|c|}{\textbf{ModelPrice}}	 \\[3pt] \hline
					\endhead
					
					\hline 
					\endlastfoot
					
					\cellcolor{LightGreen} model\_id & INT & identifikacijski broj makete \\ \hline
					\cellcolor{LightGreen} material & VARCHAR & ime materijala \\ \hline
					price & NUMERIC & cijena makete za zadani materijal \\ \hline
					
				\end{longtabu}
			
				\noindent\textbf{Order} Podaci o narudžbi koju korisnik šalje trgovini. Sadrži atribute: identifikacijski broj narudžbe, vrijeme nastanka, identifikacijski broj korisnika koji šalje narudžbu. U vezi je \textit{Many-to-One} s entitetom User i u vezi \textit{One-to-Many} s entitetom Order-Model.
				
				\begin{longtabu} to \textwidth {|X[6, l]|X[6, l]|X[20, l]|}
					
					\hline \multicolumn{3}{|c|}{\textbf{Order}}	 \\[3pt] \hline
					\endfirsthead
					
					\hline \multicolumn{3}{|c|}{\textbf{Order}}	 \\[3pt] \hline
					\endhead
					
					\hline 
					\endlastfoot
					
					\cellcolor{LightGreen} id & INT &  identifikacijski broj narudžbe; samoinkrementirajući broj jedinstven za svaku narudžbu \\ \hline
					time\_created & DATETIME &  vrijeme nastanka	\\ \hline 
					\cellcolor{LightBlue} user\_id & INT & identifikacijski broj korisnika koji šalje narudžbu \\ \hline
					
				\end{longtabu}
			
				\noindent\textbf{Order-Model} Lista maketa u narudžbi. Sadrži atribute: identifikacijski broj narudžbe, identifikacijski broj makete, materijal od kojeg je maketa izrađena i cijena makete u trenutku narudžbe.
				
				\begin{longtabu} to \textwidth {|X[6, l]|X[6, l]|X[20, l]|}
					
					\hline \multicolumn{3}{|c|}{\textbf{Order-Model}}	 \\[3pt] \hline
					\endfirsthead
					
					\hline \multicolumn{3}{|c|}{\textbf{Order-Model}}	 \\[3pt] \hline
					\endhead
					
					\hline 
					\endlastfoot
					
					\cellcolor{LightBlue} order\_id & INT &  identifikacijski broj narudžbe \\ \hline
					\cellcolor{LightBlue} model\_id & INT & identifikacijski broj makete \\ \hline
					material & VARCHAR & ime materijala od kojeg je maketa izrađena \\ \hline
					price & NUMERIC & cijena makete za zadani materijal u trenutku narudžbe \\ \hline
					
				\end{longtabu}
			
			\subsection{Dijagram baze podataka}
				\begin{figure}[H]
					\includegraphics[width=.9\linewidth]{dijagrami/ER_baza.png}
					\caption{E-R dijagram baze podataka}
					\label{fig:erdija}
				\end{figure}
			\eject
			
			
		\section{Dijagram razreda}
				
			\begin{figure}[H]
				\includegraphics[width=.9\linewidth]{slike/Dijagram_razreda.PNG} %veličina u odnosu na širinu linije
				\caption{Dijagram razreda}
				\label{fig:diraz1} %label mora biti drugaciji za svaku sliku
			\end{figure}
		
			Razredi vezani uz bazu podataka su smješteni na zasebnoj slici zbog lakše preglednosti.
		
			\begin{figure}[H]
				\includegraphics[width=.9\linewidth]{slike/Dijagram_razreda_baza.PNG} %veličina u odnosu na širinu linije
				\caption{Dijagram razreda - Baza podataka}
				\label{fig:diraz2} %label mora biti drugaciji za svaku sliku
			\end{figure}
		
		\section{Dijagram stanja}
			
			
			Dijagram stanja opisuje dinamičko ponašanje dijela sustava u vremenu. Njim se prikazuj stanje objekta te prijelazi iz jednog stanja u drugo temeljeni na događajima. Na slici ispod prikazan je dijagram stanja za prijavljenog korisnika u sustav. Korisniku se pri otvaranju stranice prikazuje početna stranica te dalje pomoću izbornika u zaglavlju pristupa funkcijama koje mu se nude. Prijavljeni korisnik može pregledavati, komentirati i predlagati priče. Također, može pregledavati standardne makete u sustavu i kupovati ih ili naručiti maketu po svojim zahtjevima. Klikom na "Korisnički račun" u izborniku korisniku se daje pregled njegovog računa sa mogučnošću izmjene podataka i vidljivosti svakog podatka. Na kraju, korisnik također može pristupiti svojoj košarici i preko nje dovršiti kupnju.
			
			\begin{figure}[H]
				\includegraphics[width=1\linewidth]{slike/Dijagram_stanja.PNG} %veličina u odnosu na širinu linije
				\caption{Dijagram stanja}
				\label{fig:dijstan} %label mora biti drugaciji za svaku sliku
			\end{figure}
			
			
			\eject 
		
		\section{Dijagram aktivnosti}
			
			\begin{figure}[H]
				\includegraphics[width=1\linewidth]{slike/Dijagram_aktivnosti_1.PNG} %veličina u odnosu na širinu linije
				\caption{Dijagram aktivnosti za predlaganje, pregled i komentiranje priča}
				\label{fig:diakt1} %label mora biti drugaciji za svaku sliku
			\end{figure}

			\begin{figure}[H]
				\includegraphics[width=1\linewidth]{slike/Dijagram_aktivnosti_2.PNG} %veličina u odnosu na širinu linije
				\caption{Dijagram aktivnosti za kupovinu i narudžbu maketa}
				\label{fig:diakt2} %label mora biti drugaciji za svaku sliku
			\end{figure}
			
			\eject
		\section{Dijagram komponenti}
		
			\textbf{\textit{dio 2. revizije}}\\
		
			 \textit{Potrebno je priložiti dijagram komponenti s pripadajućim opisom. Dijagram komponenti treba prikazivati strukturu cijele aplikacije.}